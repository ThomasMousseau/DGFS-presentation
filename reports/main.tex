%----------------------------------------------------------------------------------------
%    PACKAGES AND THEMES
%----------------------------------------------------------------------------------------

\documentclass[aspectratio=169,xcolor=dvipsnames]{beamer}
\setbeameroption{show notes} %TODO: Thomas a enlever avant la presentation
\usetheme{SimplePlus}

\useoutertheme{miniframes}  % Adds horizontal navigation dots at the top for subsections

\usecolortheme{} 

\setbeamercolor{block title}{bg=structure,fg=white}  % Navy blue background for block titles
\setbeamercolor{block body}{bg=structure!10,fg=structure}  % Light navy tint for block body


\usepackage{comment}
\usepackage{hyperref}
\usepackage{graphicx} % Allows including images
\usepackage{booktabs} % Allows the use of \toprule, \midrule and \bottomrule in tables
\usepackage{array} % Allows >{\centering\arraybackslash} in tabular

% Define hyphenation command
\newcommand{\hyp}{-}

%----------------------------------------------------------------------------------------
%    TITLE PAGE
%----------------------------------------------------------------------------------------

\title{Diffusion Generative Flow Samplers: Improving Learning Signals Through Partial Trajectory Optimization}

\subtitle{Dinghuai Zhang*, Ricky T. Q. Chen, Cheng-Hao Liu, Aaron Courville \& Yoshua Bengio}
\author{Thomas Mousseau} 

% \institute
% {
%     Department of Computer Science and Information Engineering \\
%     National Taiwan University % Your institution for the title page
% }
\date{\today} % Date, can be changed to a custom date

%----------------------------------------------------------------------------------------
%    PRESENTATION SLIDES
%----------------------------------------------------------------------------------------

\begin{document}

\begin{frame}
    % Print the title page as the first slide
    \vspace*{-2cm}
    \titlepage
\end{frame}

\begin{frame}{Overview}
    % Throughout your presentation, if you choose to use \section{} and \subsection{} commands, these will automatically be printed on this slide as an overview of your presentation
    \tableofcontents
\end{frame}

%------------------------------------------------

\section{Introduction}

\subsection{Problem Statement}

\begin{frame}[t]{Sampling from Unnormalized Densities}
\footnotesize
\textbf{Goal.} Sample from a $D$-dimensional target with unnormalized density $\mu(x)$:
\[
\pi(x)=\frac{\mu(x)}{Z},\qquad Z=\int_{\mathbb R^D}\mu(x)\,dx\ \text{(unknown)}.
\]
We assume we can evaluate $\mu(x)$ (and sometimes $\nabla \log \mu(x)$), but we have no samples from $\pi$ and do not know $Z$.

\medskip
\textbf{Context.} We seek a \emph{sampler} (peer to MCMC/VI) that produces calibrated samples and, ideally, estimates of $\log Z$, \emph{without} any dataset from $\pi$.

\vspace{0.2cm}

% \textbf{Chemistry (small molecule conformers):} For any 3D shape \(x\) we can compute an energy \(E(x)\); sampling \(\pi(x)\propto e^{-\beta E(x)}\) gives the likely shapes and how often they occur (no dataset, \(Z\) unknown).

\textbf{Chemistry (small-molecule conformers).} Use \emph{conformer strain/steric clash energy} as $E(x)$ for a 3D shape $x$ (lower when bonds/angles/rotations are comfortable). Sampling $\pi(x)\propto e^{-\beta E(x)}$ gives the \emph{population of stable rotamers} and how often each shape occurs at temperature $T$.

\vspace{0.2cm}

The integral for $Z$ over all 3D configurations is intractable, so we use the unnormalized score $\mu(x)=e^{-\beta E(x)}$, where $E(x)$ comes from chemistry/force-field physics (bonds, angles, nonbonded terms) and encodes how likely a conformation is, a lower $E$ means more likely.









\end{frame}



\begin{frame}[t]{SOC formulation (part 1/2)}
\footnotesize
\textbf{Forward (controlled) process $Q$.} A Markov chain with Gaussian transitions:
\[
Q(x_{0:N}):\quad x_0\sim p_0^{\text{ref}},\quad x_{n+1}\sim P_F(\cdot\mid x_n)=\mathcal N\!\big(x_n + h\,f(x_n,n),\; h\sigma^2 I\big).
\]

\textbf{Reference process $Q^{\text{ref}}$.} Same covariance, zero drift:
\[
x_{n+1}\sim P_F^{\text{ref}}(\cdot\mid x_n)=\mathcal N\!\big(x_n,\; h\sigma^2 I\big),\qquad x_0\sim p_0^{\text{ref}},\ \ p_n^{\text{ref}}\text{ known.}
\]

\textbf{Target process $P$.} Tie the terminal marginal to $\pi$ via the reference:
\[
P(x_{0:N})\ :=\ Q^{\text{ref}}(x_{0:N})\;\frac{\pi(x_N)}{p_N^{\text{ref}}(x_N)}.
\]
Then $P(x_N)\propto \mu(x_N)$, making $P$ a valid path-space target.
\end{frame}

\begin{frame}[t]{SOC formulation (part 2/2)}
\footnotesize

\medskip
\textbf{Learning objective (discrete-time SOC).} Learn $f$ by minimizing the path KL:
\[
\min_{f}\ D_{\mathrm{KL}}(Q\ \|\ P)
\;\;\Longleftrightarrow\;\;
\min_{f}\ \mathbb E_{Q}\!\Big[\sum_{n=0}^{N-1}\frac{h}{2\sigma^2}\,\|f(x_n,n)\|^2\;+\;\log\Psi(x_N)\Big],
\]
with \(\Psi(x_N)=\dfrac{p_N^{\text{ref}}(x_N)}{\mu(x_N)}\).
(Continuous-time limit recovers the classic VE-SDE SOC formulation.)
\end{frame}

\begin{frame}[t]{Why SOC, and what still hurts (motivation for DGFS)}
\footnotesize
\textbf{Why SOC/control-as-inference?}
\begin{itemize}\itemsep3pt
  \item Principled \emph{path-space} objective; $Z$ cancels, so only $\mu$ (and optionally $\nabla\log\mu$) is needed.
  \item Calibrated sampling by \emph{steering} a simple reference process toward the target.
\end{itemize}

\textbf{Pain point in prior SOC samplers (PIS/DDS).}
\begin{itemize}\itemsep3pt
  \item Training signal sits \emph{only at terminal time} $N$ and losses use \emph{full trajectories} $\Rightarrow$ poor credit assignment, high variance, weaker mode coverage.
\end{itemize}

\textbf{DGFS in one line.}
\begin{itemize}\itemsep3pt
  \item Keep the same SOC/path-KL setup, but introduce a learned \emph{flow function} $F_n(x_n)$ and enforce \emph{subtrajectory balance} to inject \emph{intermediate} learning signals and enable \emph{partial-trajectory} training.
\end{itemize}
\end{frame}


% \section{Introduction}

% \subsection{Problem Statement}

% \begin{frame}[t]{Problem Setting: Sampling from Unnormalized Densities}
% \footnotesize
% \textbf{Goal.} Given only an \emph{energy oracle} $E(x)$, sample from the unnormalized target
% \[
% \pi(x)\ \propto\ e^{-E(x)} \quad\text{with}\quad Z=\int e^{-E(x)}dx\ \text{(unknown)}.
% \]
% We want a \emph{sampler} (like MCMC/VI) that produces calibrated samples and, ideally, estimates of $\log Z$.

% \medskip
% \textbf{Classical paradigms we compare against.}
% \begin{itemize}\itemsep2pt
%   \item \textbf{MCMC:} Metropolis--Hastings, Langevin/HMC, SMC. \emph{Pros:} asymptotically correct, only needs $\mu=e^{-E}$. \emph{Cons:} mixing/geometry barriers.
%   \item \textbf{VI:} mean-field, \textit{normalizing flows}, amortized VI. \emph{Pros:} fast, amortizable, ELBO bound. \emph{Cons:} bias from variational family, mode-seeking.
% \end{itemize}

% \medskip
% \textbf{Our stance.} We build a \emph{sampling algorithm} inspired by diffusion models’ transport view, but \emph{not} a data-trained generator: no dataset from $\pi$, only the energy $E$.
% \end{frame}

% \begin{frame}[t]{Diffusion-inspired SOC sampler (and its pain point)}
% \footnotesize
% \textbf{Diffusion inspiration (transport by time-reversal).}
% Forward reference: $dx_t=\sigma\,dW_t$ (or VP/VE variants). Reverse-time generative dynamics:
% \[
% dx_t=\big[f(x,t)-\sigma^2\nabla_x\log p_t(x)\big]\,dt+\sigma\,d\bar W_t.
% \]
% With \emph{data}, the score $\nabla\log p_t$ is learned by DSM. \emph{Here we have no data from $\pi$}, so DSM is inapplicable.

% \medskip
% \textbf{Control-as-inference (our solution path).}
% Treat sampling as \emph{steering} a simple reference process $Q$ with control $u_\theta$:
% \[
% x_{n+1}=x_n+h\,u_\theta(n,x_n)+\sqrt{2\sigma^2 h}\,\varepsilon_n,\qquad
% dx_t=u_\theta(t,x_t)\,dt+\sigma\,dW_t.
% \]
% Train by minimizing a \emph{path-space KL} where $Z$ cancels:
% \[
% \mathcal L(\theta)=\mathbb E_{\tau\sim P_\theta}\!\big[\log P_\theta(\tau)-\log Q(\tau)-\log \mu(x_N)\big]+\text{const},
% \]
% so we can use $\mu=e^{-E}$ (and optionally $\nabla\log\mu$) directly. Samples are obtained by simulating the \emph{controlled} process.

% \medskip
% \textbf{Remaining pain point (motivates DGFS).}
% Path-KL training in prior work (PIS/DDS) delivers a \emph{single, terminal-time} signal and requires \emph{full trajectories} $\Rightarrow$ \textbf{sluggish credit assignment} and high-variance gradients.
% % (Next: DGFS injects intermediate supervision via a learned flow and subtrajectory balance.)
% \end{frame}



% \begin{frame}[t]{Why view diffusion as SOC / Control-as-Inference?}
% \footnotesize
% \textbf{Goal.} Turn sampling from unnormalized $\pi(x)\propto\mu(x)$ into \emph{steering} a simple diffusion toward $\mu$ with a principled objective.

% \medskip
% \textbf{Objective (path-space KL):}
% \[
% \min_\theta\ \mathrm{KL}\!\Big(P_\theta(\tau)\,\Big\|\,Q(\tau)\tfrac{\mu(x_N)}{Z}\Big)
% = \mathbb E_{P_\theta}\!\big[\log P_\theta(\tau)-\log Q(\tau)-\log\mu(x_N)\big]+\mathrm{const}.
% \]

% \textbf{Benefits.}
% \begin{itemize}\itemsep3pt
%   \item \textbf{No partition function:} $Z$ cancels inside the KL.
%   \item \textbf{Temporal decomposition:} induce local/Bellman or flow-balance constraints $\Rightarrow$ intermediate signals, better credit assignment.
%   \item \textbf{Off-policy friendly:} optimize a path measure; exploration controlled via KL-to-reference.
%   \item \textbf{Theory leverage:} links to Schr\"odinger bridges and entropy-regularized OT for guarantees and algorithms.
% \end{itemize}

% \textbf{DGFS angle.} Add a learned flow $F_n$ to enforce \emph{subtrajectory} balance, bringing GFlowNet-style intermediate supervision into diffusion control.
% \end{frame}


\subsection{Limitations of Prior Work}

\begin{frame}[t]{Prior Work \& Limitations}
\textbf{SOC framing via SDEs.} Representing diffusion models as controlled SDEs lets us cast sampling as a stochastic optimal control (SOC) or control-as-inference problem (minimize a KL over \emph{path measures}).

\medskip
\textbf{Path Integral Sampler (PIS).}
\begin{itemize}
  \item Uses Schr\"odinger bridge / path-integral control to learn $u_\theta$ that transports a reference diffusion to match $\pi$ at time $N$.
  \item Principled \emph{trajectory-space} objective that works with \emph{unnormalized} $\mu$; supports off-policy training.
  \item \textit{Limitation:} Objective requires \emph{full trajectories} and rewards at terminal time only \(\Rightarrow\) sluggish credit assignment, higher gradient variance.
\end{itemize}

\textbf{Denoising Diffusion Samplers (DDS).}
\begin{itemize}
  \item Also recast sampling as control over the forward path; improves practical sampling and stability vs. vanilla diffusion heuristics.
  \item \textit{Limitation:} Still optimized in \emph{trajectory space} with \emph{terminal-time} learning signals; inherits credit-assignment and variance issues.
\end{itemize}

\medskip
\textbf{Takeaway.} Moving to the trajectory space gives us \underline{a principled KL objective over paths and compatibility with unnormalized targets}, but at the cost of \underline{terminal-time credit assignment \& high-variance training signals}. This motivates adding \emph{intermediate} supervision and \emph{partial-trajectory} learning (what DGFS provides).
\end{frame}


%------------------------------------------------
\section{Methodology}

\subsection{DGFS framework}

\begin{frame}{DGFS Core}
    \begin{block}{Trajectories with Learned Flow $F_n$}
        DGFS uses trajectories guided by a learned flow $F_n$ for sampling.
    \end{block}
    \begin{itemize}
        \item Partial-trajectory updates via SubTB (Subtrajectory Balance).
        \item Intermediate training signals for better credit assignment.
        \item Controlled forward process and subtrajectory balance constraint.
        \item Partition-function estimator for evaluation.
        \item Two nets: Drift f and scalar flow $F_n$.
        \item Training setup, step embeddings, and cost vs. inference speed trade-offs.
    \end{itemize}
\end{frame}

\subsection{Why It Works}

\begin{frame}{Why It Works}
    \begin{block}{Evidence for Effectiveness}
        DGFS reduces gradient variance and improves stability.
    \end{block}
    \begin{itemize}
        \item Lower gradient variance through partial trajectories.
        \item Better mode coverage and stability in sampling.
        \item Compatibility with off-policy exploration.
    \end{itemize}
\end{frame}

%------------------------------------------------
\section{Results and Limitations}

\subsection{Results}

\begin{frame}{Results}
    \begin{block}{Key Findings}
        Evaluation of log Z bias across various targets.
    \end{block}
    \begin{itemize}
        \item Visual grids for distributions like MoG (Mixture of Gaussians) and Manywell.
        \item Key ablations on training components and hyperparameters.
    \end{itemize}
\end{frame}

\subsection{Limitations \& Open Questions}

\begin{frame}{Limitations \& Open Questions}
    \begin{block}{Challenges and Future Work}
        Designing intermediate signals remains an open problem.
    \end{block}
    \begin{itemize}
        \item Scaling to high-dimensional and costly $\mu$ (target distributions).
        \item Fixed backward policy $P_B$ limits flexibility.
    \end{itemize}
\end{frame}

%------------------------------------------------
\section{Conclusion}

\subsection{Key Insights}
% Add content: Summarize DGFS's impact and paper's contributions.

\subsection{Future Directions and Usage since its release}
% Add content: Discuss limitations, extensions, and personal thoughts.

\begin{frame}{Conclusion}

\end{frame}


\end{document}