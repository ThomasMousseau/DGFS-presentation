%----------------------------------------------------------------------------------------
%    PACKAGES AND THEMES
%----------------------------------------------------------------------------------------

\documentclass[aspectratio=169,xcolor=dvipsnames]{beamer}
\setbeameroption{show notes} %TODO: Thomas a enlever avant la presentation
\usetheme{SimplePlus}

\useoutertheme{miniframes}  % Adds navigation dots at the top for subsections

\usecolortheme{} 

\setbeamercolor{block title}{bg=structure,fg=white}  % Navy blue background for block titles
\setbeamercolor{block body}{bg=structure!10,fg=structure}  % Light navy tint for block body


\usepackage{comment}
\usepackage{hyperref}
\usepackage{graphicx} % Allows including images
\usepackage{booktabs} % Allows the use of \toprule, \midrule and \bottomrule in tables
\usepackage{array} % Allows >{\centering\arraybackslash} in tabular

%----------------------------------------------------------------------------------------
%    TITLE PAGE
%----------------------------------------------------------------------------------------

\title{Diffusion Generative Flow Samplers: Improving Learning Signals Through Partial Trajectory Optimization}

\subtitle{Dinghuai Zhang*, Ricky T. Q. Chen, Cheng-Hao Liu, Aaron Courville \& Yoshua Bengio}
\author{Thomas Mousseau} 

% \institute
% {
%     Department of Computer Science and Information Engineering \\
%     National Taiwan University % Your institution for the title page
% }
\date{\today} % Date, can be changed to a custom date

%----------------------------------------------------------------------------------------
%    PRESENTATION SLIDES
%----------------------------------------------------------------------------------------

\begin{document}

\begin{frame}
    % Print the title page as the first slide
    \vspace*{-2cm}
    \titlepage
\end{frame}

\begin{frame}{Overview}
    % Throughout your presentation, if you choose to use \section{} and \subsection{} commands, these will automatically be printed on this slide as an overview of your presentation
    \tableofcontents
\end{frame}

%------------------------------------------------
\section{Problem Setup}

\subsection{Legged Locomotion Challenges}

\begin{frame}{Legged Locomotion Challenges}
    \begin{block}{Emerging Topic in Legged Robotics}
        Terrain-aware locomotion has become an emerging topic in legged robotics. However, it is hard to generate diverse, challenging, and realistic unstructured terrains in simulation, which limits the way researchers evaluate their locomotion policies.
    \end{block}

\end{frame}

\subsection{GFlowNet's Role in Addressing Terrain Benchmarking Challenges}
% Add content: Discuss current benchmarks' shortcomings and the need for generative methods.

\begin{frame}{GFlowNet's Role in Addressing Terrain Benchmarking Challenges}
    \begin{block}{Key Challenges for Terrain Benchmarking}
        To achieve reliable quantification of robustness, terrain samples should resemble real terrains in the wild. The generation process should be controllable to produce high-quality terrains that are challenging to a user-specified extent. Finally, the sampler must maintain terrain quality and diversity simultaneously.
    \end{block}
    \begin{itemize}
        \item GFlowNet ensures high-quality terrains and diversity among those included in the final benchmark by providing probabilistic sampling from unnormalized distributions.
        \item It addresses realism through flow-based modeling of complex terrains.
        \item Controllability via reward-guided active learning for user-specified challenges.
        \item Quality-diversity balance through efficient exploration of terrain spaces.
    \end{itemize}
\end{frame}
%------------------------------------------------
\section{Methodology and Architecture}

\subsection{Conditional GAN for Terrain Generation}
% Add content: Explain procedural terrain generation and parameters.

\subsection{GFlowNet and Active Learning}
% Add content: Detail GFlowNet's role in sampling, flow matching, and the active learning loop.

%------------------------------------------------

%------------------------------------------------

\section{Evaluations and Results}

\subsection{Diversity and Quality of Generated Terrains}
% Add content: Describe simulation environments, metrics, and robot models.

\subsection{Sample Efficiency and Benchmarking Performance}
% Add content: Present benchmarks, performance comparisons, and ablation studies.

%------------------------------------------------

%------------------------------------------------
\section{Conclusion}

\subsection{Key Insights}
% Add content: Summarize GFlowNet's impact and paper's contributions.

\subsection{Future Directions and Usage since its release}
% Add content: Discuss limitations, extensions, and personal thoughts.

\begin{frame}{Conclusion}

\end{frame}


\end{document}