%----------------------------------------------------------------------------------------
%    PACKAGES AND THEMES
%----------------------------------------------------------------------------------------

\documentclass[aspectratio=169,xcolor=dvipsnames]{beamer}
\setbeameroption{show notes} %TODO: Thomas a enlever avant la presentation
\usetheme{SimplePlus}
\useoutertheme{miniframes}  % Adds navigation dots at the top for subsections
\usecolortheme{default}  % Adds a green color scheme; adjust for darker green if needed

\definecolor{darkgreen}{RGB}{0,50,0}  % Adjust RGB values for desired shade (e.g., darker: 0,30,0)

\setbeamercolor{structure}{fg=darkgreen}  % For titles, frames, etc.
\setbeamercolor{normal text}{fg=darkgreen}  % For body text

\usepackage{comment}
\usepackage{hyperref}
\usepackage{graphicx} % Allows including images
\usepackage{booktabs} % Allows the use of \toprule, \midrule and \bottomrule in tables
\usepackage{array} % Allows >{\centering\arraybackslash} in tabular

%----------------------------------------------------------------------------------------
%    TITLE PAGE
%----------------------------------------------------------------------------------------

\title{Generating a Terrain-Robustness Benchmark for Legged Locomotion:
A Prototype via Terrain Authoring and Active Learning}
\subtitle{Chong Zhang and Lizhi Yang}

\author{Thomas Mousseau} 

% \institute
% {
%     Department of Computer Science and Information Engineering \\
%     National Taiwan University % Your institution for the title page
% }
\date{\today} % Date, can be changed to a custom date

%----------------------------------------------------------------------------------------
%    PRESENTATION SLIDES
%----------------------------------------------------------------------------------------

\begin{document}

\begin{frame}
    % Print the title page as the first slide
    \vspace*{-2cm}
    \titlepage
\end{frame}

\begin{frame}{Overview}
    % Throughout your presentation, if you choose to use \section{} and \subsection{} commands, these will automatically be printed on this slide as an overview of your presentation
    \tableofcontents
\end{frame}

%------------------------------------------------
\section{Problem Setup}

\subsection{Legged Locomotion Challenges}
% Add content: Introduce robotics context, terrain variability, and real-world examples.

\subsection{Benchmark Limitations}
% Add content: Discuss current benchmarks' shortcomings and the need for generative methods.

%------------------------------------------------
\section{Methodology and Architecture}

\subsection{Terrain Authoring}
% Add content: Explain procedural terrain generation and parameters.

\subsection{Active Learning with GFlowNet}
% Add content: Detail GFlowNet's role in sampling, flow matching, and the active learning loop.

%------------------------------------------------

%------------------------------------------------

\section{Experiments and Results}

\subsection{Experimental Setup}
% Add content: Describe simulation environments, metrics, and robot models.

\subsection{Results and Analysis}
% Add content: Present benchmarks, performance comparisons, and ablation studies.

%------------------------------------------------

%------------------------------------------------
\section{Conclusion}

\subsection{Key Insights}
% Add content: Summarize GFlowNet's impact and paper's contributions.

\subsection{Future Directions}
% Add content: Discuss limitations, extensions, and personal thoughts.

\begin{frame}{Conclusion}

\end{frame}


\end{document}